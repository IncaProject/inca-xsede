
%
%  $Description: Author guidelines and sample document in LaTeX 2.09$ 
%
%  $Author: ienne $
%  $Date: 1995/09/15 15:20:59 $
%  $Revision: 1.4 $
%

\documentclass[times, 10pt,twocolumn]{article} 
\usepackage{latex8}
\usepackage{times}

%\documentstyle[times,art10,twocolumn,latex8]{article}

%------------------------------------------------------------------------- 
% take the % away on next line to produce the final camera-ready version 
\pagestyle{empty}

%------------------------------------------------------------------------- 
\begin{document}

\title{User-level Grid Monitoring with Inca 2}

\author{Shava Smallen, Kate Ericson, and Jim Hayes \\
San Diego Supercomputer Center\\ University of California, San Diego\\ 
9500 Gilman Drive, La Jolla, CA 92093-0505, USA\\ 
\{ssmallen,kericson,jhayes\}@sdsc.edu\\
}

\maketitle
\thispagestyle{empty}

\begin{abstract}
User-level Grid monitoring is valuable, how it relates to other Grid
monitoring,  and what are its implementation challenges.
Inca 1 good but had limitations.  In this paper, we introduce Inca 2 and
describe its features and architecture.  We then show a few use cases for Inca
is being deployed on TeraGrid and XXX?  System impact results and performance
results.  Finally, we discuss our future work.  
\end{abstract}

% Goals:
% 1. Motivated and differentiate the Grid monitoring Inca does and 
% 3. Describe Inca 2 design and its benefits
% 4. Illustrate that Inca 2 design is mature and being used in production by
%     several Grids
% 2. Not to restate stuff from Inca 1 paper

%------------------------------------------------------------------------- 
\Section{Introduction}

Grid monitoring is x, y, and z.  Describe the stakeholders.

User-level Grid monitoring is x and is valuable.  

However, implementing user-level Grid monitoring has challenges.  List.

Inca 1~\cite{inca1} first addressed this but had limitations.  We learned
lessons from our TeraGrid deployment.  This motivated our development of a new
version of Inca.


%------------------------------------------------------------------------- 
\Section{Design and Implementation}

Features list.

Overview of all architecture components -- arch picture


%------------------------------------------------------------------------- 

\SubSection{Inca Components}


%\begin{figure}[h]
%   \caption{Example of caption.}
%\end{figure}
%
%\noindent Long captions should be set as in 
%\begin{figure}[h] 
%   \caption{Example of long caption requiring more than one line. It is 
%     not typed centered but aligned on both sides and indented with an 
%     additional margin on both sides of 1~pica.}
%\end{figure}

%------------------------------------------------------------------------- 
\Section{Use Cases}

\SubSection{TeraGrid}

\SubSection{DEISA??}

%------------------------------------------------------------------------- 
\Section{System Impact and Performance Results}

Maybe.

%------------------------------------------------------------------------- 
\Section{Summary}

%------------------------------------------------------------------------- 
\bibliographystyle{latex8}
\bibliography{paper}

\end{document}

